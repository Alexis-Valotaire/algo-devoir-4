\documentclass[devoir4.tex]{subfiles}

\begin{document}

\section*{Question 1}
Donnez toutes les étapes de l'algorithme de Miller-Rabin pour tester si 97 est premier. Vous devez supposer que le générateur pseudo aléatoire retourne le nombre 5.
\newline

Ce que nous savons du problème:
\begin{align*}
	n &= 97 \\
	n-1 &= 96 = 2^5 * 3 \\
	b &= 5
\end{align*}

On peut alors exécuter l'algorithme de Miller-Robin pour déterminer si 97 est possiblement un nombre premier:
\begin{align*}
	b^t = 5^3 &= 125 \equiv 28 (mod \:97) \\
	b^{2t} = 28^2 &= 784 \equiv 8 (mod \:97) \\
	b^{2^2t} = 8^2 &= 64 \equiv 64 (mod \:97) \\
	b^{2^3t} = 64^2 &= 4096 \equiv 22 (mod \:97) \\
	b^{2^4t} = 22^2 &= 484 \equiv 96 (mod \:97) \equiv -1 (mod \:97)
\end{align*}

Puisque nous avons une valeur de -1 à la dernière itération montrée, nous pouvons donc dire que 97 est probablement premier sur une base 5.

\end{document}
