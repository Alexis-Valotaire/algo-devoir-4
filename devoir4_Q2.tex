\documentclass[devoir4.tex]{subfiles}

\begin{document}

\section*{Question 2}
On peut montrer que si \(n\) est un nombre premier, alors pour tout entier positif \(b\) on a :

\begin{equation}
	b^{(n-1)/2} mod \: n = J(b,n)
	\label{eqQ21}
\end{equation}

où \(J(b, n)\) est une fonction appelée symbole de Jacobi (Il n'est pas nécessaire de connaître la définition exacte de \(J(b, n)\) pour répondre à cette question). \\

D'autre part, si n est composé, alors la relation (\ref{eqQ21}) est fausse pour au moins 50\% de tous les entiers b pour lesquels \(PGCD(b, n) = 1\). \\

\textbf{a)} En utilisant ces deux observations, trouvez un algorithme de Monte Carlo 50\%-correct qui détermine si un entier positif n est premier. Vous pouvez supposer qu’il est possible de tester efficacement si la relation (\ref{eqQ21}) est vraie. \\ 

\textbf{b)} Votre algorithme est-il biaisé? Si oui, est-il vrai-biaisé ou faux biaisé? Expliquez. \\

\textbf{c)} Montrez comment vous pouvez modifier votre algorithme pour en obtenir un qui soit 99.999\%-correct.

\end{document}
