\documentclass[devoir4.tex]{subfiles}

\begin{document}

\section*{Question 2}
On peut montrer que si \(n\) est un nombre premier, alors pour tout entier positif \(b\) on a :

\begin{equation}
	b^{(n-1)/2} mod \: n = J(b,n)
	\label{eqQ21}
\end{equation}

où \(J(b, n)\) est une fonction appelée symbole de Jacobi (Il n'est pas nécessaire de connaître la définition exacte de \(J(b, n)\) pour répondre à cette question). \\

D'autre part, si n est composé, alors la relation (\ref{eqQ21}) est fausse pour au moins 50\% de tous les entiers b pour lesquels \(PGCD(b, n) = 1\). \\

\textbf{a)} En utilisant ces deux observations, trouvez un algorithme de Monte Carlo 50\%-correct qui détermine si un entier positif n est premier. Vous pouvez supposer qu’il est possible de tester efficacement si la relation (\ref{eqQ21}) est vraie. \\ 

\begin{algorithm}[H]
	\KwData{\( entier \: n, entier \: k \)}
    \KwResult{Vrai si n est premier ou Faux si n est composé}
   
    \vspace{0.2cm}
	\For{\( i \gets 1 $ \KwTo $ k \)}
    {
		\(b \gets uniforme(2 \dots n-2) \)
	
		\If{ \(PGCD(b,n) = 1\)}{
	    		\If{ \(b^{(n-1)/2} mod \: n \neq J(b,n)\)}{
			    	\Return Faux
				}
	    	}
	}
	
	\Return Vrai
	
      \caption{VerifSiPremierMonteCarlo50Correct} 
\end{algorithm}

\vspace{0.4cm}

\textbf{b)} Votre algorithme est-il biaisé? Si oui, est-il vrai-biaisé ou faux-biaisé? Expliquez. \\

Notre algorithme est biaisé puisqu'il possède une probabilité de 50\% ce qui est la limite inférieure de la condition pour être considéré comme biaisé. De plus, il est faux-biaisé car nous sommes sûr d'avoir un \(n\) composé lorsque la fonction retourne \(Faux\), contrairement à avoir un \(n\) premier où nous avons 50\% de chance d'avoir un faux-vrai.

\newpage

\textbf{c)} Montrez comment vous pouvez modifier votre algorithme pour en obtenir un qui soit 99.999\%-correct.

\vspace{0.2cm}

\begin{algorithm}[H]
	\KwData{\( entier \: n, entier \: k \)}
    \KwResult{Vrai si n est premier ou Faux si n est composé}
   
    \(temoin \gets 0 \) 
   
    \vspace{0.2cm}
	\For{\( i \gets 1 $ \KwTo $ k \)}
    {
		\(b \gets uniforme(2 \dots n-2) \)
	
		\If{ \(PGCD(b,n) = 1\)}{
	    		\If{ \(b^{(n-1)/2} mod \: n \neq J(b,n)\)}{
					\(temoin++\)			    	
				}
	    	}
	}
	\If{\(\frac{temoin}{k} \cdot 100 \geq 50 \)}{
		\Return Faux
	} \Else{
		\Return Vrai
	}
	
      \caption{VerifSiPremierMonteCarlo99Correct} 
\end{algorithm}

\end{document}
